\documentclass[12pt]{article}
%%---------------------------------------------------------------------
% packages
% geometry
\usepackage{geometry}
% font
\usepackage{fontspec}
\defaultfontfeatures{Mapping=tex-text}  %%如果没有它,会有一些 tex 特殊字符无法正常使用,比如连字符。
\usepackage{xunicode,xltxtra}
\usepackage[BoldFont,SlantFont,CJKnumber,CJKchecksingle]{xeCJK}  % \CJKnumber{12345}: 一万二千三百四十五
\usepackage{CJKfntef}  %%实现对汉字加点、下划线等。
\usepackage{pifont}  % \ding{}
% math
\usepackage{amsmath,amsfonts,amssymb}
% color
\usepackage{color}
\usepackage{xcolor}
\definecolor{EYE}{RGB}{199,237,204}
\definecolor{FLY}{RGB}{128,0,128}
\definecolor{ZHY}{RGB}{139,0,255}
% graphics
\usepackage[americaninductors,europeanresistors]{circuitikz}
\usepackage{tikz}
\usetikzlibrary{positioning,arrows,shadows,shapes,calc,mindmap,trees,backgrounds}  % placements=positioning
\usepackage{graphicx}  % \includegraphics[]{}
\usepackage{subfigure}  %%图形或表格并排排列
% table
\usepackage{colortbl,dcolumn}  %% 彩色表格
\usepackage{multirow}
\usepackage{multicol}
\usepackage{booktabs}
% code
\usepackage{fancyvrb}
\usepackage{listings}
% title
\usepackage{titlesec}
% head/foot
\usepackage{fancyhdr}
% ref
\usepackage{hyperref}
% pagecolor
\usepackage[pagecolor={EYE}]{pagecolor}
% tightly-packed lists
\usepackage{mdwlist}

\usepackage{styles/iplouccfg}
\usepackage{styles/zhfontcfg}
\usepackage{styles/iplouclistings}

%%---------------------------------------------------------------------
% settings
% geometry
\geometry{left=2cm,right=1cm,top=2cm,bottom=2cm}  %设置 上、左、下、右 页边距
\linespread{1.5} %行间距
% font
\setCJKmainfont{Adobe Kaiti Std}
%\setmainfont[BoldFont=Adobe Garamond Pro Bold]{Apple Garamond}  % 英文字体
%\setmainfont[BoldFont=Adobe Garamond Pro Bold,SmallCapsFont=Apple Garamond,SmallCapsFeatures={Scale=0.7}]{Apple Garamond}  %%苹果字体没有SmallCaps
\setCJKmonofont{Adobe Fangsong Std}
% graphics
\graphicspath{{figures/}}
\tikzset{
    % Define standard arrow tip
    >=stealth',
    % Define style for boxes
    punkt/.style={
           rectangle,
           rounded corners,
           draw=black, very thick,
           text width=6.5em,
           minimum height=2em,
           text centered},
    % Define arrow style
    pil/.style={
           ->,
           thick,
           shorten <=2pt,
           shorten >=2pt,},
    % Define style for FlyZhyBall
    FlyZhyBall/.style={
      circle,
      minimum size=6mm,
      inner sep=0.5pt,
      ball color=red!50!blue,
      text=white,},
    % Define style for FlyZhyRectangle
    FlyZhyRectangle/.style={
      rectangle,
      rounded corners,
      minimum size=6mm,
      ball color=red!50!blue,
      text=white,},
    % Define style for zhyfly
    zhyfly/.style={
      rectangle,
      rounded corners,
      minimum size=6mm,
      ball color=red!25!blue,
      text=white,},
    % Define style for new rectangle
    nrectangle/.style={
      rectangle,
      draw=#1!50,
      fill=#1!20,
      minimum size=5mm,
      inner sep=0.1pt,}
}
\ctikzset{
  bipoles/length=.8cm
}
% code
\lstnewenvironment{VHDLcode}[1][]{%
  \lstset{
    basicstyle=\footnotesize\ttfamily\color{black},%
    columns=flexible,%
    framexleftmargin=.7mm,frame=shadowbox,%
    rulesepcolor=\color{blue},%
%    frame=single,%
    backgroundcolor=\color{yellow!20},%
    xleftmargin=1.2\fboxsep,%
    xrightmargin=.7\fboxsep,%
    numbers=left,numberstyle=\tiny\color{blue},%
    numberblanklines=false,numbersep=7pt,%
    language=VHDL%
    }\lstset{#1}}{}
\lstnewenvironment{VHDLmiddle}[1][]{%
  \lstset{
    basicstyle=\scriptsize\ttfamily\color{black},%
    columns=flexible,%
    framexleftmargin=.7mm,frame=shadowbox,%
    rulesepcolor=\color{blue},%
%    frame=single,%
    backgroundcolor=\color{yellow!20},%
    xleftmargin=1.2\fboxsep,%
    xrightmargin=.7\fboxsep,%
    numbers=left,numberstyle=\tiny\color{blue},%
    numberblanklines=false,numbersep=7pt,%
    language=VHDL%
    }\lstset{#1}}{}
\lstnewenvironment{VHDLsmall}[1][]{%
  \lstset{
    basicstyle=\tiny\ttfamily\color{black},%
    columns=flexible,%
    framexleftmargin=.7mm,frame=shadowbox,%
    rulesepcolor=\color{blue},%
%    frame=single,%
    backgroundcolor=\color{yellow!20},%
    xleftmargin=1.2\fboxsep,%
    xrightmargin=.7\fboxsep,%
    numbers=left,numberstyle=\tiny\color{blue},%
    numberblanklines=false,numbersep=7pt,%
    language=VHDL%
    }\lstset{#1}}{}
% pdf
\hypersetup{%pdfpagemode=FullScreen,%
            pdfauthor={Haiyong Zheng},%
            pdftitle={Title},%
            CJKbookmarks=true,%
            bookmarksnumbered=true,%
            bookmarksopen=false,%
            plainpages=false,%
            colorlinks=true,%
            citecolor=green,%
            filecolor=magenta,%
            linkcolor=cyan,%red(default)
            urlcolor=cyan}
% section
%http://tex.stackexchange.com/questions/34288/how-to-place-a-shaded-box-around-a-section-label-and-name
\newcommand\titlebar{%
\tikz[baseline,trim left=3.1cm,trim right=3cm] {
    \fill [cyan!25] (2.5cm,-1ex) rectangle (\textwidth+3.1cm,2.5ex);
    \node [
        fill=cyan!60!white,
        anchor= base east,
        rounded rectangle,
        minimum height=3.5ex] at (3cm,0) {
        \textbf{\thesection.}
    };
}%
}
\titleformat{\section}{\Large\bf\color{blue}}{\titlebar}{0.1cm}{}
% head/foot
\setlength{\headheight}{15pt}
\pagestyle{fancy}
\fancyhf{}

\chead{\color{black!50!green}Assignment \#3}

%\lfoot{\color{blue!50!green}Dai Jialun}
\cfoot{\color{blue!50!green}\href{http://vision.ouc.edu.cn/~zhenghaiyong}{CVBIOUC}}
\rfoot{\color{blue!50!green}$\cdot$\ \thepage\ $\cdot$}
\renewcommand{\headrulewidth}{0.4pt}
\renewcommand{\footrulewidth}{0.4pt}

%%---------------------------------------------------------------------
\begin{document}
%%---------------------------------------------------------------------
%%---------------------------------------------------------------------
% \titlepage
\title{\vspace{-2em} 浮游动物识别分类——深度学习\\
\normalsize{}}
\author{Dai Jialun \hspace{0.25in} Wu Bin}
\date{\vspace{-0.7em} \today \vspace{-0.7em}}
%%---------------------------------------------------------------------
\maketitle\thispagestyle{fancy}
%%---------------------------------------------------------------------
\maketitle
%\tableofcontents 
\section{问题}
深度学习使用的数据库基本都是ImageNet,因此使用ImageNet与ZooScan的训练样本进行比较。
\begin{description}
\item[小数据] ZooScan的训练样本有13类(包含负样本),总共有9460张图片。ImageNet的ILSVRC 2013的训练样本有200类,总共395909张图片。
\item[分类] 将浮游动物分为不同种类,且种类数目较少,只有10种。
\item[数据的标注] ZooScan的训练图像中只有单个物体,即single-label,且背景为空白。ImageNet的训练图像基本都用多个不同物体的标注,即multiple-label。
\end{description}
\begin{figure}[!ht]
  \centering 
  \subfigure[]{ 
  %  \label{fig: } %% label for first subfigure 
    \includegraphics[width=0.32\textwidth]{ILSVRC}} 
  \subfigure[]{ 
  %  \label{fig: result1: b} %% label for second subfigure 
    \includegraphics[width=0.35\textwidth]{ILSVRC2013}} 
  \caption{ILSVRC 2013}
%  \label{fig: } %% label for entire figure 
\end{figure}

\begin{figure}[!ht]
  \centering 
  \subfigure[]{ 
  %  \label{fig: } %% label for first subfigure 
    \includegraphics[width=0.185\textwidth]{zooplankton1}} 
  \subfigure[]{ 
 %   \label{fig: result1: b} %% label for second subfigure 
    \includegraphics[width=0.2\textwidth]{zooplankton2}} 
  \caption{Zooplankton}
%  \label{fig: } %% label for entire figure 
\end{figure}


\section{方案}
\subsection{开源软件}
\begin{description}
\item[Caffe] Caffe是一个清晰而高效的深度学习框架,由表达式、速度和模块化组成,其作者是博士毕业于UC Berkeley的贾扬清。它具有有开放性架构、扩展代码、速度与社区化的优点。Caffe是纯粹的C++/CUDA架构,支持命令行、Python和MATLAB接口,而且可以在CPU和GPU直接无缝切换。

\item[cxxnet] 一个机器学习项目,其具有轻量而齐全的框架、强大统一的并行计算接口和易于扩展的代码结构。cxxnet平衡了python为主的编程效率和以c++为核心的为代表的追逐性能,使得我们在不损失效率的前提下可以通过模板编程技术允许开发者编写和matlab/numpy类似的代码,并且在编译时自动展开成优化的kernel。另外一些特性:CuDNN的支持、及时更新的最新技术、Caffe模型转换和方便的语言接口。

\item[Cuda-convnet] Cuda-convnet基于C++/CUDA编写,采用反向传播算法的深度卷积神经网络实现。Cuda-convnet 是 Alex Krizhevsky 公开的一套CNN代码,运行于Linux系统上,使用GPU做运算;2014年发布的版本可以支持多GPU上的数据并行和模型并行训练。(据说Alex的这套代码还是不太友好,按理说应该给出一个IO的标准,供后来人测试其他数据集的。)

\item[Theano] 提供了在深度学习数学计算方面的Python库,让你可以有效定义、优化和评价包含多维数组的数学公式。它整合了NumPy矩阵计算库、可以运行在GPU上、提供良好的算法上的扩展、在优化方面进行加速和稳定与动态C代码的生成等。

\item[OverFeat] 由纽约大学CILVR实验室开发的基于卷积神经网络系统,主要应用场景为图像识别和图像特征提取。主要使用C++编写的库来运行OverFeat卷积网络,可用数据增强来实现提高分类结果。

\item[Torch7] 一个为机器学习算法提供广泛支持的科学计算框架,其中的神经网络工具包(Package)实现了均方标准差代价函数、非线性激活函数和梯度下降训练神经网络的算法等基础模块,可以方便地配置出目标多层神经网络开展训练实验。
\end{description}

\subsection{网络模型}
在图像识别与分类上,主要使用卷积神经网络(Convolutional Neural Network, CNN)的网络模型来实现的。
\begin{description}
\item [AlexNet] AlexNet\cite{krizhevsky2012imagenet}是Hinton与其学生为了回应别人对于deep learning的质疑而将deep learning用于ImageNet的ILSVRC 2012上,结果优于当时最优的水平。该模型训练了一个深度卷积神经网络来完成分类和检测任务。AlexNet包含5层卷积层和3层全连接层。
\begin{figure}[!ht]
\centering
\includegraphics[width=0.55\textwidth]{AlexNet}
\caption{AlexNet}
%\label{fig:framework}
\end{figure}

\item[VGGNet] VGGNet\cite{simonyan2013deep}是Visual Geometry Group,University of Oxford在ILSVRC 2013分类上使用的模型,结合了两种深度框架:the deep Fisher vector network 和 the deep convolutional network。
\begin{figure}[!ht]
\centering
\includegraphics[width=0.3\textwidth]{vggnet1}
\caption{VGGNet}
%\label{fig:framework}
\end{figure}

\item[OverFeat] OverFeat\cite{sermanet2013overfeat}%\footnote{OverFeat: Integrated Recognition, Localization and Detection using Convolutional Networks}
由CILVR Lab,New York University开发的,基于卷积网络的图像识别和特征提取系统。
OverFeat的选举网络是由ImageNet 1K数据集训练的,参加了ILSVRC 2013分类、定位与检测竞赛,应用了在卷积网络上的多尺度和扫描框方法。
\begin{figure}[!ht]
\centering
\includegraphics[width=0.5\textwidth]{overfeat}
\caption{overfeat}
%\label{fig:framework}
\end{figure}

\item[NIN] NIN(Network In Network)\cite{Lin:2013aa}是Learn­ing and Vi­sion Re­search Group,	Na­tion­al Uni­ver­si­ty of Sin­ga­pore所使用的深度学习框架。其中,他们团队提出``adaptive non-parametric rectification''方法来提高准确率。 
\begin{figure}[!ht]
\centering
\includegraphics[width=0.5\textwidth]{NIN}
\caption{NIN}
%\label{fig:framework}
\end{figure}

\item[GoogLeNet] GoogLeNet\cite{simonyan2014very}是Google团队在ILSVRC 2014上,使用的深度学习网络模型。
\begin{figure}[!ht]
\centering
\includegraphics[width=0.5\textwidth]{googlenet}
\caption{GoogLeNet}
%\label{fig:framework}
\end{figure}

\item[VGGNet2] VGGNet2\cite{szegedy2014going}是Visual Geometry Group在ILSVRC 2014上,使用的深度学习网络模型。
\begin{figure}[!ht]
\centering
\includegraphics[width=0.5\textwidth]{vggnet2}
\caption{VGGNet2}
%\label{fig:framework}
\end{figure}

\item[PCANet] PCANet\footnote{PCANet: A Simple Deep Learning Baseline for Image Classification?}是一个基于CNN的简化Deep Learning模型。经典的CNN存在的问题是参数训练时间过长且需要特别的调参技巧。PCANet模型是一种训练过程简单,且能适应不同任务、不同数据类型的网络模型。卷积核是直接通过PCA计算得到的,而不是像CNN一样通过反馈迭代得到的。
\begin{figure}[!ht]
\centering
\includegraphics[width=0.5\textwidth]{PCANet}
\caption{PCANet}
%\label{fig:framework}
\end{figure}

\item[MP-CNN]  
\end{description}

\section{优化技巧}
\subsection{Dropout}
应用在全连接层,是一种引入噪声机制,避免过拟合的有效正规化方法。适用于样本数目数 $\ll$ 模型参数。对于每一个隐层的output,以50\%的概略将它们设置为0,不再对与forward或backward过程起作用,即迫使每个神经元不依赖某些特定神经元独立工作,从而学到更有用的特征。
\begin{figure}[!ht]
  \centering 
  \subfigure[]{ 
 %   \label{fig: } %% label for first subfigure 
    \includegraphics[width=0.25\textwidth]{dropout}} 
  \subfigure[]{ 
 %   \label{fig: result1: b} %% label for second subfigure 
    \includegraphics[width=0.4\textwidth]{dropout1}} 
  \caption{ILSVRC 2013}
 % \label{fig: } %% label for entire figure 
\end{figure}


\subsection{Pooling}

用更高层的抽象表示图像特征, 降维,CNN重点,不同的pooling框不重合,表示分辨率的下降,信息的融合。
\begin{figure}[!ht]
\centering
\includegraphics[width=0.5\textwidth]{pooling}
\caption{Pooling}
%\label{fig:framework}
\end{figure}

\subsection{PCA}
Principal Components Analysis,主成分分析。主要使用在降维,对每个RGB图像进行一个PCA变换,完成去噪功能与保证图像多样性,加入了一个随机尺度因子,每一轮重新生成一个尺度因子,保证了同一副图像中在显著特征上有一定范围的变换,降低overfitting。

\subsection{Image transformation}
通过对于图像的变换实现了对于数据集合的扩大,在位置的维度上丰富了训练数据。降低了overfitting和网络结构设计的复杂程度。

\subsection{LPN}
Local Response Normalization,局部响应归一化。完成“临近抑制”操作,对局部输入区域进行归一化。在通道间归一化模式中,局部区域范围在相邻通道间,没有空间扩展。在通道内归一化模式中,局部区域在空间上扩展,只针对独立通道进行。

\subsection{Pre-training}
``逐层初始化''(Layer-wise Pre-training)。采用的逐层贪婪训练方法来训练网络的参数,,即先训练网络的第一个隐含层,然后接着训练第二个,第三个。最后用这些训练好的网络参数值作为整体网络参数的初始值。前几层的网络层次基本都用无监督的方法容易获得,只有最后一个输出层需要有监督的数据。
\begin{figure}[!ht]
\centering
\includegraphics[width=0.5\textwidth]{pre-training}
\caption{pre-training}
%\label{fig:framework}
\end{figure}

\subsection{优化参数算法}
SGD:随机梯度下降法。


\section{方法}
\subsection{Caffe}
Caffe(Convolutional Architecture for Fast Feature Embedding)是纯粹的C++/CUDA架构,支持命令行、Python和MATLAB接口,且在CPU和GPU直接无缝切换。
\subsubsection{Caffe优点:}
\begin{itemize}
\item 上手快:模型与相应优化都是以文本形式而非代码形式给出。
\item 速度快:能够运行较好的模型与海量的数据。
\item 模块化:方便扩展到新的任务和设置上。
\item 开放性:公开的代码和参考模型用于再现。
\item 社区好:可以通过BSD-2参与开发与讨论。
\end{itemize}




\subsubsection{代码层次}
\begin{description}
\item[\textbf{Blob}] 基础的数据结构,{\color{blue} 保存学习到的参数}以及{\color{blue} 网络传输过程中产生数据的类}。\\
	主要是对protocol buffer所定义的数据结构的继承,Caffe也因此可以在尽可能小的内存占用下获得很高的效率。在更高一级的Layer中Blob用下面的形式表示学习到的参数:
	\begin{itemize}
	\item vector<sharedptr<Blob<Dtype> > > blob
	\item vector<Blob<Dtype>*> \& bottom
	\item vector<Blob<Dtype>*> *top
	\end{itemize}
\item[\textbf{Layer}] 网络的基本单元,由此派生出了各种层类。修改这部分的人主要是{\color{blue} 研究特征表达}方向的。\\
	用某种Layer来表示卷积操作,非线性变换,Pooling,权值连接等操作。具体分为5大类Layer:
	\begin{itemize}
	\item \textbf{NeuronLayer} \hspace{0.1in} 定义于neuron\_layers.hpp中,其派生类主要是元素级别的运算,运算均为同址计算。
	\item \textbf{LossLayer} \hspace{0.1in} 定义于loss\_layers.hpp中,其派生类会产生loss,只有这些层能够产生loss。
	\item \textbf{数据层} \hspace{0.1in} 定义于data\_layer.hpp中,作为网络的最底层,主要实现数据格式的转换。
	\item \textbf{特征表达层} \hspace{0.1in} 定义于vision\_layers.hpp,实现特征表达功能,例如卷积操作,Pooling操作等。
	\item \textbf{网络连接层和激活函数(待定)} \hspace{0.1in} 定义于common\_layers.hpp,Caffe提供了单个层与多个层的连接,并在这个头文件中声明。这里还包括了常用的全连接层InnerProductLayer类。
	\end{itemize}
	在Layer内部,数据主要有两种传递方式,\textbf{正向传导(Forward)}和\textbf{反向传导(Backward)}。Caffe中所有的Layer都要用这两种方法传递数据。
\item[\textbf{Net}] 网络的搭建,{\color{blue} 将Layer所派生出层类组合成网络}。\\
Net用容器的形式将多个Layer有序地放在一起,其自身实现的功能主要是对逐层Layer进行初始化,以及提供Update()的接口(更新网络参数),本身不能对参数进行有效地学习。
	\begin{itemize}
	\item vector<shared\_ptr<Layer<Dtype> > > layers \_
	\item vector<Blob<Dtype>*> \& Forward()
	\item void Net<Dtype>::Backward()
	\end{itemize}
\item[\textbf{Solver}] Net的求解,修改这部分人主要会是{\color{blue} 研究DL求解}方向的。\\
	这个类中包含一个Net的指针,主要是实现了训练模型参数所采用的优化算法,它所派生的类就可以对整个网络进行训练了。
	\begin{itemize}
	\item shared\_ptr<Net<Dtype> > net\_
	\item virtual void ComputeUpdateValue() = 0
	\end{itemize}
\end{description}

\subsection{CNN}
Deep Learning是全部深度学习算法的总称,CNN是深度学习算法在图像处理领域的一个应用。

\subsubsection{CNN优点:}
\begin{itemize}
\item 权值共享网络结构,降低网络模型的复杂度,减少了权值的数量。
\item 图像直接作为网络的输入,避免复杂的特征提取和数据重建。
\item 对平移、比例缩放、倾斜或者其他形式的变形具有高度不变性。
\end{itemize}

\subsubsection{CNN组成:}
\begin{itemize}
\item 局部感知
\item 参数共享
\item 多卷积核
\item Down-pooling
\item 多层卷积
\end{itemize}


\subsubsection{CNN网络配置文件:}
\begin{itemize}
\item Imagenet\_solver.prototxt (包含全局参数的配置文件)
\item Imagenet.prototxt (包含训练网络的配置文件)
\item Imagenet\_val.prototxt (包含测试网络的配置文件)
\end{itemize}



\subsection{AlexNet}
\subsubsection{网络结构}
AlexNet网络结构是CNN类型的DeepLearning网络模型在图像分类上的应用,如图Figure~\ref{fig:framework}。

\begin{figure}[!ht]
\centering
\includegraphics[width=0.8\textwidth]{AlexNet}
\caption{AlexNet}
\label{fig:framework}
\end{figure}

该模型训练了一个深度卷积神经网络,来将ILSVRC-2010中1.2M的高分辨率图像数据分为1000类。测试结果,Top-1和Top-5的错误率分别为37.5\%和17\%。优于当时最优的水平。后来作者利用该种模型的变体参与了ILSVRC-2012比赛,以Top-5错误率15.3\%遥遥领先亚军的26.2\%。该神经网络包含60M参数和650K神经元,用5个卷积层(其中某些层与亚采样层连接)、三个全连接层(包括一个1K门的输出层)。为使训练更快,文章采用非饱和神经元,包括了大量不常见和新的特征来提升性能,减少训练时间。并利用了一个高效的GPU应用进行卷积运算。最终网络包含5层卷积层和3层全连接层。而这个层深度很重要,因为移除任何一个卷积层,将会导致更差的结果。网络大小主要受限于GPU的内存和训练时间。实验证明,本网络在有两个GTX 580 3GB GPU的机器上训练了5-6天。实验结果显示,如果GPU更快或数据集更大,实验结果将会更好。架构上的改进:
	\begin{itemize}
	\item ReLU非线性特征 
	\item 在多GPU上训练
	\item 局部响应标准化
	\item 重叠采样
	\end{itemize}
\textbf{全局架构}:网络包括八个有权值的层。前五层是卷积层,剩下的三层是全连接的。最后一个全连接层是输出层。第二、四、五个卷积层只与上层在同一个GPU上的Kernel Map(下文称特征图)连接,第三个卷积层与第二个卷积层的所有特征图连接。全连接层的神经元与上一层的所有神经元连接(即改进二)。响应标准化层在第一和第二个卷积层之后(改进三),降采样层在响应标准化层和第五个卷积层之后(改进四)。而ReLU非线性公式在每个卷积层和全连接层都有应用(改进一)。
\textbf{降低过拟合}:
\begin{itemize}
\item 数据增强:通过提取图片的5个224*224切片(中间和四角)和它们的水平翻转来做出预测,预测结果为十次预测的平均值。第二种数据增强的方式为改变训练图像RGB通道的强度,对RGB空间做PCA,然后对主成分做一个(0, 0.1)的高斯扰动。结果让错误率又下降了百分一。

\item Dropout:将某些层隐藏,按照50\%的概率输出0。这些隐藏的神经元不会参加CNN的forward过程,也不会参加back propagation过程。测试中,我们在前两个全连接层使用了该方法,利用他们所有的神经元,但其输出都乘以了0.5。没有dropout,我们的网络将会产生严重的过拟合。
\end{itemize}

每移走一层中间的层,Top-1的错误率将会增高2\%。

但为了简化实验,文章没有使用任何无监督学习。如果那样,那么在更大规模的网络和更长时间训练的情况下,我们得结果就能得到提高。

结果显示,大规模、低深度的CNN在监督学习下能够达到破纪录的好结果。而本文章的模型的层数不能随意变动,实验证明,每移走一层中间的层,Top-1的错误率将会增高2%。

虽然如果我们能够有更好的计算性能来增大网络规模,却不增加标签数据,可能会提高最后的正确率,但为了简化实验,文章没有使用任何无监督学习。如果那样,那么在更大规模的网络和更长时间训练的情况下,我们得结果就能得到提高。但我们仍然有很多需求来进行时空下视觉系统的研究。最终我们就能用很大很深的深度卷积网络来处理视频序列,以得到静态图片可能会忽视的信息。

\begin{itemize}
\item 输入224$\times$224的图片,3通道
\item 第一层卷积 + pooling:11$\times$11的卷积核96个,步长为4,每个GPU各有48个。max-pooling的核为2$\times$2。如图Figure~\ref{fig:conv1}。
	\begin{figure}[!ht]
	\centering
	\includegraphics[width=0.7\textwidth]{conv1_1}
	\caption{Conv1}
	\label{fig:conv1}
	\end{figure}
\end{itemize}

\begin{itemize}
\item 第二层卷积 + pooling:5$\times$5的卷积核256个,每个GPU各有128个。max-pooling的核为2$\times$2。如图Figure~\ref{fig:conv2}。
	\begin{figure}[!ht]
	\centering
	\includegraphics[width=0.7\textwidth]{conv2_2}
	\caption{Conv2}
	\label{fig:conv2}
	\end{figure}
\end{itemize}

\begin{itemize}
\item 第三层卷积:与上一层是全连接。3$\times$3的卷积核384个,每个GPU各有192个。如图Figure~\ref{fig:conv3}。
	\begin{figure}[!ht]
	\centering
	\includegraphics[width=0.7\textwidth]{conv3}
	\caption{Conv3}
	\label{fig:conv3}
	\end{figure}
\item 第四层卷积:3$\times$3的卷积核384个,每个GPU各有192个。如图Figure~\ref{fig:conv4}。
	\begin{figure}[!ht]
	\centering
	\includegraphics[width=0.7\textwidth]{conv4}
	\caption{Conv4}
	\label{fig:conv4}
	\end{figure}
\item 第五层卷积 + pooling:3$\times$3的卷积核256个,每个GPU各有128个。max-pooling:2 $\times$2的核。如图Figure~\ref{fig:conv5}。
	\begin{figure}[!ht]
	\centering
	\includegraphics[width=0.7\textwidth]{conv4}
	\caption{Conv5}
	\label{fig:conv5}
	\end{figure}
\end{itemize}

\begin{itemize}
\item 第六层全连接:4096维,将第五层的输出连接成为一个一维向量,作为该层的输入。如图Figure~\ref{fig:fc6}。
	\begin{figure}[!ht]
	\centering
	\includegraphics[width=0.6\textwidth]{fc1}
	\caption{Fc6}
	\label{fig:fc6}
	\end{figure}
\item 第七层全连接:4096维,将第六层的输出连接成为一个一维向量。如图Figure~\ref{fig:fc7}。
	\begin{figure}[!ht]
	\centering
	\includegraphics[width=0.6\textwidth]{fc2}
	\caption{Fc7}
	\label{fig:fc7}
	\end{figure}
\item Softmax层:输出为1000,输出的每一维都是图片属于该类别的概率。如图Figure~\ref{fig:fc8}。
	\begin{figure}[!ht]
	\centering
	\includegraphics[width=0.6\textwidth]{fc3}
	\caption{Fc8}
	\label{fig:fc8}
	\end{figure}
\end{itemize}

\bibliographystyle{plain}
\bibliography{DeepLearning}

%%---------------------------------------------------------------------
\end{document}
