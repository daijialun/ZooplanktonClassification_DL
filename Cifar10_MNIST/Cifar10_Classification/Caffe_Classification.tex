%插入样式内容
\input{Initial}
%%---------------------------------------------------------------------
\begin{document}
%%---------------------------------------------------------------------
%%---------------------------------------------------------------------
% \titlepage
\title{\vspace{-2em} 使用Caffe进行Cifar10图像分类的流程\\
\normalsize{}}
\author{Wu Bin \hspace{0.25in} Dai Jialun}
\date{\vspace{-0.7em} \today \vspace{-0.7em}}
%%---------------------------------------------------------------------
\maketitle\thispagestyle{fancy}
%%---------------------------------------------------------------------
\maketitle
%\tableofcontentsG
\section{创建训练集}
在测试阶段我们选择的训练集是CIFAR-10,共有60000张图像,其中有10000张为测试图像,这个分为10类,详细的介绍在\href{https://github.com/daijialun/ZooplanktonClassification\_DL/tree/master/Cifar10\_MNIST}{GitHub} 上有详细的介绍。

其实在拿到一个数据集后,一般是包含两部分的,一是图像本身,二是图像中包含label数据。有了训练集以后我们要把它转换成Caffe可以读取的数据格式,主要有Lmdb和LevelDB两种形式。

先说LevelDB,这是一个由Google实现的非常高效的Key-Value数据库,支持十亿级别的数据量。主要归功于LSM算法,使得它在非常大的数量级别下还有非常高的性能。

LMDB的全称是Lightning Memory-Mapped Database,闪电般的内存映射数据库。它文件结构简单,一个文件夹,里面一个数据文件,一个锁文件。数据随意复制,随意传输。它的访问简单,不需要运行单独的数据库管理进程,只要在访问数据的代码里引用LMDB库,访问时给文件路径即可。

它们都是键/值对(Key/Value Pair)嵌入式数据库管理系统编程库。虽然lmdb的内存消耗是leveldb的1.1倍,但是lmdb的速度比leveldb快10\%至15\%,更重要的是lmdb允许多种训练模型同时读取同一组数据集。因此lmdb取代了leveldb成为Caffe默认的数据集生成格式。

对于CIFAR-10数据集,Caffe通过其example下的“create\_cifar10.sh”脚本对data目录下的二进制的Cifar数据集转换成lmdb格式。其中主要用到了“convert\_cifar\_data.cpp”这个程序,这个程序目前还没有看明白。
当然如果数据集是自己的数据集的话,可以通过caffe\_root/examples/imagenet下的“create\_imagenet.sh”脚本将自有的图片和Label文件转化为lmdb格式,不过这个目前还在测试中。



\section{对训练集进行训练}

在选定CIFAR-10作为训练集的情况下, 可以直接使用Caffe提供的model进行训练,CIFAR-10 模型是一个由卷积层、池层、修正线性单元和一个在最顶层的局部对比度归一化的线性分类器组成的卷积神经网络。并且这个模型已经在 CAFFE\_ROOT/examples/cifar10下的“cifar10\_quick\_train\_test.prototxt”和“cifar10\_full\_train\_test.prototxt”给定了。

在训练的过程中,只要使用对应的model对测试集进行训练就好了。

\begin{lstlisting}
cd $CAFFE_ROOT
./examples/cifar10/train_full.sh
\end{lstlisting}

通过阅读“train\_full.sh”,可以知道脚本主要使用了 \mcode{caffe train --solver=examples/cifar10/cifar10_full_solver.prototxt}

%%---------------------------------------------------------------------
\end{document}
